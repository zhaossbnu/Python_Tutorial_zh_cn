\chapter{虚拟环境和包\label{VirtualEnv}}
\markboth{虚拟环境和包}{}
\section{简介}
Python 应用程序经常会使用一些不属于标准库的包和模块。应用程序有时候需要某个特定版本的库,因为它需要一个特定的 bug 已得到修复的库或者它是使用了一个过时版本的库的接口编写的。

这就意味着可能无法安装一个 Python 来满足每个应用程序的要求。如果应用程序 A 需要一个特定模块的 1.0 版本但是应用程序 B 需要该模块的 2.0 版本,这两个应用程序的要求是冲突的,安装版本 1.0 或者版本 2.0 将会导致其中一个应用程序不能运行。

这个问题的解决方案就是创建一个 \href{https://docs.python.org/3/glossary.html#term-virtual-environment}{\textit{虚拟环境}} (通常简称为 “virtualenv”),包含一个特定版本的 Python,以及一些附加的包的独立的目录树。

不同的应用程序可以使用不同的虚拟环境。为了解决前面例子中的冲突,应用程序 A 可以有自己的虚拟环境,其中安装了特定模块的 1.0 版本。而应用程序 B 拥有另外一个安装了特定模块 2.0 版本的虚拟环境。如果应用程序 B 需求一个库升级到 3.0 的话,这也不会影响到应用程序 A 的环境。
\section{创建虚拟环境}
用于创建和管理虚拟环境的脚本叫做 \texttt{pyvenv}。\texttt{pyvenv} 通常会安装你可用的 Python 中最新的版本。这个脚本也能指定安装一个特定的版本的 Python,因此如果在你的系统中有多个版本的 Python 的话,你可以运行 \texttt{pyvenv-3.5} 或者你想要的任何版本来选择一个指定的 Python 版本。

要创建一个 virtualenv,首先决定一个你想要存放的目录接着运行 \texttt{pyvenv} 后面携带着目录名:

\verb|pyvenv tutorial-env|

如果目录不存在的话,这将会创建一个 \texttt{tutorial-env} 目录,并且也在目录里面创建一个包含 Python 解释器,标准库,以及各种配套文件的 Python “副本”。

一旦你已经创建了一个虚拟环境,你必须激活它。

在 Windows 上,运行:

\verb|tutorial-env/Scripts/activate|

在 Unix 或者 MacOS 上,运行:

\verb|source tutorial-env/bin/activate|

(这个脚本是用 bash shell 编写的。如果你使用 \texttt{csh} 或者 \texttt{fish} shell,你应该使用 \texttt{activate.csh} 和 \texttt{activate.fish} 来替代。)

激活了虚拟环境会改变你的 shell 提示符,显示你正在使用的虚拟环境,并且修改了环境以致运行 \texttt{python} 将会让你得到了特定的 Python 版本。例如:
\begin{lstlisting}
-> source ~/envs/tutorial-env/bin/activate
(tutorial-env) -> python
Python 3.5.2+ (3.4:c7b9645a6f35+, May 22 2015, 09:31:25)
  ...
>>> import sys
>>> sys.path
['', '/usr/local/lib/python35.zip', ...,
'~/envs/tutorial-env/lib/python3.5/site-packages']
>>>
\end{lstlisting}
\section{使用pip管理包}
一旦你激活了一个虚拟环境,可以使用一个叫做 \texttt{pip} 程序来安装,升级以及删除包。默认情况下 \texttt{pip} 将会从 Python Package Index,\textless \url{https://pypi.python.org/pypi}\textgreater, 中安装包。你可以通过 web 浏览器浏览它们,或者你也能使用 \texttt{pip} 有限的搜索功能:
\begin{lstlisting}
(tutorial-env) -> pip search astronomy
skyfield               - Elegant astronomy for Python
gary                   - Galactic astronomy and gravitational dynamics.
novas                  - The United States Naval Observatory NOVAS astronomy library
astroobs               - Provides astronomy ephemeris to plan telescope observations
PyAstronomy            - A collection of astronomy related tools for Python.
...
\end{lstlisting}
\texttt{pip} 有许多子命令:“搜索”,“安装”,“卸载”,“freeze”(译者注:这个词语暂时没有合适的词语来翻译),等等。(请参考 installing-index 指南获取 \texttt{pip} 更多完整的文档。)

你可以安装一个包最新的版本,通过指定包的名称:
\begin{Verbatim}[fontfamily=tt]
-> pip install novas
Collecting novas
  Downloading novas-3.1.1.3.tar.gz (136kB)
Installing collected packages: novas
  Running setup.py install for novas
Successfully installed novas-3.1.1.3
\end{Verbatim}
你也能安装一个指定版本的包,通过给出包名后面紧跟着 \verb|==| 和版本号:
\begin{Verbatim}[fontfamily=tt]
-> pip install requests==2.6.0
Collecting requests==2.6.0
  Using cached requests-2.6.0-py2.py3-none-any.whl
Installing collected packages: requests
Successfully installed requests-2.6.0
\end{Verbatim}
如果你重新运行命令,\texttt{pip} 会注意到要求的版本已经安装,不会去做任何事情。你也可以提供一个不同的版本号来安装,或者运行 \texttt{pip install --upgrade} 来升级包到最新版本:
\begin{Verbatim}[fontfamily=tt]
-> pip install --upgrade requests
Collecting requests
Installing collected packages: requests
  Found existing installation: requests 2.6.0
    Uninstalling requests-2.6.0:
      Successfully uninstalled requests-2.6.0
Successfully installed requests-2.7.0
\end{Verbatim}
\texttt{pip uninstall} 后跟一个或者多个包名将会从虚拟环境中移除这些包。

\texttt{pip show} 将会显示一个指定的包的信息:
\begin{Verbatim}[fontfamily=tt]
(tutorial-env) -> pip show requests
---
Metadata-Version: 2.0
Name: requests
Version: 2.7.0
Summary: Python HTTP for Humans.
Home-page: http://python-requests.org
Author: Kenneth Reitz
Author-email: me@kennethreitz.com
License: Apache 2.0
Location: /Users/akuchling/envs/tutorial-env/lib/python3.4/site-packages
Requires:
\end{Verbatim}
\texttt{pip list} 将会列出所有安装在虚拟环境中的包:
\begin{Verbatim}[fontfamily=tt]
(tutorial-env) -> pip list
novas (3.1.1.3)
numpy (1.9.2)
pip (7.0.3)
requests (2.7.0)
setuptools (16.0)
\end{Verbatim}
\texttt{pip freeze} 将会生成一个类似需要安装的包的列表,但是输出采用了 \texttt{pip install} 期望的格式。常见的做法就是把它们放在一个 \texttt{requirements.txt} 文件:
\begin{Verbatim}[fontfamily=tt]
(tutorial-env) -> pip freeze > requirements.txt
(tutorial-env) -> cat requirements.txt
novas==3.1.1.3
numpy==1.9.2
requests==2.7.0
\end{Verbatim}
\texttt{requirements.txt} 能够被提交到版本控制中并且作为一个应用程序的一部分。用户们可以使用 \texttt{install -r} 安装所有必须的包:
\begin{Verbatim}[fontfamily=tt]
-> pip install -r requirements.txt
Collecting novas==3.1.1.3 (from -r requirements.txt (line 1))
  ...
Collecting numpy==1.9.2 (from -r requirements.txt (line 2))
  ...
Collecting requests==2.7.0 (from -r requirements.txt (line 3))
  ...
Installing collected packages: novas, numpy, requests
  Running setup.py install for novas
Successfully installed novas-3.1.1.3 numpy-1.9.2 requests-2.7.0
\end{Verbatim}

\texttt{pip} 还有更多的选项。请参考 installing-index 指南获取关于 \texttt{pip} 完整的文档。当你编写一个包并且在 Python Package Index 中也出现的话,请参考 distributing-index 指南。