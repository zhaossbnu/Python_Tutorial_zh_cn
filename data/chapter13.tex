\chapter{接下来?}
\markboth{接下来?}{} 读过这本指南应该会让你有兴趣使用 Python —— 可能你已经期待着用 Python 解决你的实际问题了。可以在哪里进行一步学习?

入门指南是 Python 文档集的一部分。其中的另一些文档包括:
\begin{compactitem}
    \item Python 标准库:

        应该浏览一下这份文档,它为标准库中的类型、函数和模块提供了完整(尽管很简略)的参考资料。标准的 Python 发布版包括了 大量 的附加模块。其中有针对读取 Unix 邮箱、接收 HTTP 文档、生成随机数、解析命令行选项、写 CGI 程序、压缩数据以及很多其它任务的模块。略读一下库参考会给你很多解决问题的思路。
    \item 安装 Python 模块 展示了如何安装其他 Python 用户编写的附加模块。
    \item Python 语言参考: 詳細说明了 Python 语法和语义。

它读起来很累,不过对于语言本身,有份完整的手册很有用。
\end{compactitem}
其它 Python 资源:
\begin{compactitem}
    \item \url{http://www.python.org}: Python 官方网站。它包含代码、文档和 Web 上与 Python 有关的页面链接该网站镜像于全世界的几处其它问题,类似欧洲、日本和澳大利亚。

        镜像可能会比主站快,这取决于你的地理位置。
    \item \url{http://docs.python.org}: 快速访问 Python 的文档。
    \item \url{http://pypi.python.org}: Python 包索引,以前昵称为奶酪店,索引了可供下载的,用户创建的 Python 模块。如果你发布了代码,可以注册到这里,这样别人可以找到它。
    \item \url{http://code.activestate.com/recipes/langs/python/}: Python 食谱是大量的示例代码、大型的集合,和有用的脚本。

        值得关注的是这次资源已经结集成书,名为《Python 食谱》(O’Reilly \& Associates, ISBN 0-596-00797-3。)
    \item \url{http://scipy.org}: The Scientific Python 项目包括数组快速计算和处理模块,和大量线性代数、傅里叶变换、非线性solvers、随机数分布,统计分析以及类似的包

        它读起来很累,不过对于语言本身,有份完整的手册很有用。
\end{compactitem}
与 Python 有关的问题,以及问题报告,可以发到新闻组 \texttt{comp.lang.python} ,或者发送到邮件组 \url{mailto:python-list@python.org}。新闻组和邮件组是开放的,所以发送的消息可以自动的跟到另一个之后。每天有超过 120 个投递(高峰时有数百),提问(以及回答)问题,为新功能提建议,发布新模块。在发信之前,请查阅 常见问题 (亦称 FAQ),或者在 Python 源码发布包的 Misc/ 目录中查阅。邮件组也可以在 \url{http://mail.python.org/pipermail/} 访问。FAQ回答了很多被反复提到的问题,很可能已经解答了你的问题。