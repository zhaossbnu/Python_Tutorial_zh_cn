\chapter{使用Python解释器\label{Interpreter}}
\markboth{使用Python解释器}{}
\section{调用Python解释器}
Python解释器通常被安装在目标机器的\texttt{/usr/local/bin/python3.5}目录下。将 \texttt{/usr/local/bin}目录包含进 Unix shell 的搜索路径里,以确保可以通过输入:

\verb|python3.5|\\
命令来启动他\footnote{在 Unix 系统上,Python 3.X 解释器默认未被安装成名为\texttt{python}的命令,所以它不会与同时安装在系统中的 Python 2.x 命令冲突。}。 由于 Python 解释器的安装路径是可选的,这也可能是其它路径,你可以联系安装 Python 的用户或系统管理员确认(例如,\texttt{/usr/local/python}就是一个常见的选择)。

在 Windows 机器上,Python 通常安装在 \texttt{C:\textbackslash{}Python35} 位置,当然你可以在运行安装向导时修改此值。要想把此目录添加到你的 PATH 环境变量中,你可以在 DOS 窗口中输入以下命令:

\texttt{set path=\%path\%;C:\textbackslash{}python35}

通常你可以在主窗口输入一个文件结束符(Unix 系统是 \texttt{Control-D},Windows 系统是 \texttt{Control-Z})让解释器以 0 状态码退出。如果那没有作用,你可以通过输入\texttt{quit()}命令退出解释器。

Python 解释器具有简单的行编辑功能。在 Unix 系统上,任何 Python 解释器都可能已经添加了 GNU readline 库支持,这样就具备了精巧的交互编辑和历史记录等功能。在 Python 主窗口中输入 \texttt{Control-P} 可能是检查是否支持命令行编辑的最简单的方法。如果发出嘟嘟声(计算机扬声器),则说明你可以使用命令行编辑功能;更多快捷键的介绍请参考\Nameref{EditAndHistory}。如果没有任何声音,或者显示 \texttt{\^{}P}字符,则说明命令行编辑功能不可用;你只能通过退格键从当前行删除已键入的字符并重新输入。

Python 解释器有些操作类似 Unix shell:当使用终端设备(tty)作为标准输入调用时,它交互的解释并执行命令;当使用文件名参数或以文件作为标准输入调用时,它读取文件并将文件作为 脚本 执行。

第二种启动 Python 解释器的方法是\texttt{python -c command [arg] ...},这种方法可以在 命令行 执行 Python 语句,类似于 shell 中的\texttt{-c} 选项。由于 Python 语句通常会包含空格或其他特殊 shell 字符,一般建议将 命令 用单引号包裹起来。

有一些 Python 模块也可以当作脚本使用。你可以使用\texttt{python -m module [arg] ...}命令调用它们,这类似在命令行中键入完整的路径名执行模块源文件一样。

使用脚本文件时,经常会运行脚本然后进入交互模式。这也可以通过在脚本之前加上\texttt{-i} 参数来实现。
\subsection{参数传递}
调用解释器时,脚本名和附加参数传入一个名为 \texttt{sys.argv}的字符串列表。你能够获取这个列表通过执行 \texttt{import sys},列表的长度大于等于1;没有给定脚本和参数时,它至少也有一个元素:\texttt{sys.argv[0]} 此时为空字符串。脚本名指定为 ‘\verb|-|’ (表示标准输入)时, \texttt{sys.argv[0]} 被设定为‘\verb|-|’,使用 \texttt{-c}指令 时,\texttt{sys.argv[0]}被设定为‘\verb|-c|’。使用 \texttt{-m}模块 参数时,\texttt{sys.argv[0]} 被设定为指定模块的全名。 \texttt{-c}指令 或者 \texttt{-m} 模块 之后的参数不会被 Python 解释器的选项处理机制所截获,而是留在 \texttt{sys.argv} 中,供脚本命令操作。
\subsection{交互模式}
从 tty 读取命令时,我们称解释器工作于\textit{交互模式}。这种模式下它根据\textit{主提示符}来执行,主提示符通常标识为三个大于号(\verb|>>>|);继续的部分被称为\textit{从属提示符},由三个点标识(\verb|...|)。在第一行之前,解释器打印欢迎信息、版本号和授权提示:
\begin{lstlisting}
$ python3.5
Python 3.5.2 (default, Mar 16 2014, 09:25:04)
[GCC 4.8.2] on linux
Type "help", "copyright", "credits" or "license" for more information.
>>>
\end{lstlisting}

输入多行结构时需要从属提示符了,例如,下面这个 \texttt{if}语句:
\begin{lstlisting}
>>> the_world_is_flat = True
>>> if the_world_is_flat:
...     print("Be careful not to fall off!")
...
Be careful not to fall off!
\end{lstlisting}
关于交互模式更多的内容,请参见\Nameref{Appendix:InteractiveMode}。
\section{解释器及其环境}
\subsection{源程序编码}
默认情况下,Python源文件是 UTF-8 编码。在此编码下,全世界大多数语言的字符可以同时用在字符串、标识符和注释中 ——尽管 Python 标准库仅使用 ASCII 字符做为标识符,这只是任何可移植代码应该遵守的约定。如果要正确的显示所有的字符,你的编辑器必须能识别出文件是 UTF-8 编码,并且它使用的字体能支持文件中所有的字符。

你也可以为源文件指定不同的字符编码。为此,在行(首行)后插入至少一行特殊的注释行来定义源文件的编码:
\begin{lstlisting}
# -*- coding: encoding -*-
\end{lstlisting}

通过此声明,源文件中所有的东西都会被当做用encoding指代的编码来代替UTF-8。在 Python 库参考手册\texttt{codecs}一节中你可以找到一张可用的编码列表。

例如,如果你的编辑器不支持UTF-8编码的文件,但支持像Windows-1252的其他一些编码,你可以定义:
\begin{lstlisting}
# -*- coding: cp-1512 -*-
\end{lstlisting}

这样就可以在源文件中使用 Windows--1252 字符集中的所有字符了。这个特殊的编码注释必须在文件中的\emph{第一或第二行}定义。