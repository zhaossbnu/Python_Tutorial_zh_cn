\chapter{附录\label{Appendix}}
\markboth{附录}{}
\section{交互模式\label{Appendix:InteractiveMode}}
\subsection{错误处理}
当错误发生时,解释器打印一个错误信息和堆栈跟踪。在交互模式下,它返回主提示符;当输入来自文件的时候,在打印堆栈跟踪后以非零退出状态退出。(在 \texttt{try} 声明中被 except 子句捕捉到的异常在这种情况下不是错误。)有些错误是非常致命的会导致一个非零状态的退出;这也适用于内部错误以及某些情况的内存耗尽。所有的错误信息都写入到标准错误流;来自执行的命令的普通输出写入到标准输出。

输入中断符(通常是 \texttt{Control-C} 或者 \texttt{DEL})到主或者从提示符中慧取消输入并且返回到主提示。\footnote{GNU 的 Readline 包的问题可能会阻止这种做法。} 当命令执行中输入中断符会引起 \texttt{KeyboardInterrupt} 异常,这个异常能够被一个 \texttt{try} 声明处理。
\subsection{可执行 Python 脚本}
在 BSD’ish Unix 系统上,Python 脚本可直接执行,像 shell 脚本一样,只需要把下面内容加入到
\begin{lstlisting}
#!/usr/bin/env python3.5
\end{lstlisting}
(假设 python 解释器在用户的 \texttt{PATH} 中)脚本的开头,并给予该文件的可执行模式。\verb|#|! 必须是文件的头两个字符。在一些系统上,第一行必须以 Unix-style 的行结束符('\verb|\n|')结束,不能以 Windows 的行结束符('\verb|\r\n|')。 注意 '\verb|#|' 在 Python 中是用于注释的。

使用 \texttt{chmod} 命令能够给予脚本执行模式或者权限。

\verb|$ chmod +x myscript.py|

在 Windows 系统上,没有一个 “可执行模式” 的概念。Python 安装器会自动地把 \texttt{.py} 文件和 \texttt{python.exe} 关联起来,因此双击 Python 分拣将会把它当成一个脚本运行。文件扩展名也可以是 \texttt{.pyw},在这种情况下,运行时不会出现控制台窗口。
\subsection{交互式启动文件}
当你使用交互式 Python 的时候,它常常很方便地执行一些命令在每次解释器启动时。你可以这样做:设置一个名为 \texttt{PYTHONSTARTUP} 的环境变量为包含你的启动命令的文件名。这跟 Unix shells 的 \texttt{.profile} 特点有些类似。

这个文件在交互式会话中是只读的,在当 Python 从脚本中读取命令,以及在当 \texttt{/dev/tty} 被作为明确的命令源的时候不只是可读的。该文件在交互式命令被执行的时候在相同的命名空间中能够被执行,因此在交互式会话中定义或者导入的对象能够无需授权就能使用。你也能在文件中更改提示 \texttt{sys.ps1} 和 \texttt{sys.ps2}。

如果你想要从当前目录中读取一个附加的启动文件,你可以在全局启动文件中编写代码像这样:\texttt{if os.path.isfile('.pythonrc.py'): exec(open('.pythonrc.py').read())}。如果你想要在脚本中使用启动文件的话,你必须在脚本中明确要这么做:
\begin{lstlisting}
import os
filename = os.environ.get('PYTHONSTARTUP')
if filename and os.path.isfile(filename):
    with open(filename) as fobj:
       startup_file = fobj.read()
    exec(startup_file)
\end{lstlisting}
\subsection{定制模块}
Python 提供两个钩子为了让你们定制 \texttt{sitecustomize} 和 {usercustomize}。为了看看它的工作机制的话,你必须首先找到你的用户 site-packages 目录的位置。启动 Python 并且运行这段代码:
\begin{lstlisting}
>>> import site
>>> site.getusersitepackages()
'/home/user/.local/lib/python3.4/site-packages'
\end{lstlisting}
现在你可以创建一个名为 \texttt{usercustomize.py} 的文件在你的用户 site-packages 目录,并且在里面放置你想要的任何内容。它会影响 Python 的每一次调用,除非它以 \texttt{-s} (禁用自动导入)选项启动。

\texttt{sitecustomize} 以同样地方式工作,但是通常由是机器的管理员创建在全局的 site-packages 目录中,并且是在 \texttt{usercustomize} 之前导入。请参阅 \texttt{site} 模块获取更多信息。