\chapter{错误和异常\label{ErrorException}}
\markboth{错误和异常}{}
至今为止还没有进一步的谈论过错误信息,不过在你已经试验过的那些例子中,可能已经遇到过一些。Python 中(至少)有两种错误:\textit{语法错误}和\textit{异常}。
\section{语法错误}
语法错误,也被称作解析错误,也许是你学习 Python 过程中最常见抱怨:
\begin{lstlisting}
>>> while True print('Hello world')
  File "<stdin>", line 1, in ?
    while True print('Hello world')
                   ^
SyntaxError: invalid syntax
\end{lstlisting}
语法分析器指出错误行,并且在检测到错误的位置前面显示一个小“箭头”。 错误是由箭头 前面 的标记引起的(或者至少是这么检测的): 这个例子中,函数 \texttt{print()} 被发现存在错误,因为它前面少了一个冒号(‘\verb|:|’)。 错误会输出文件名和行号,所以如果是从脚本输入的你就知道去哪里检查错误了。
\section{异常}
即使一条语句或表达式在语法上是正确的,当试图执行它时也可能会引发错误。运行期检测到的错误称为 异常,并且程序不会无条件的崩溃:很快,你将学到如何在 Python 程序中处理它们。然而,大多数异常都不会被程序处理,像这里展示的一样最终会产生一个错误信息:
\begin{lstlisting}
>>> 10 * (1/0)
Traceback (most recent call last):
  File "<stdin>", line 1, in ?
ZeroDivisionError: int division or modulo by zero
>>> 4 + spam*3
Traceback (most recent call last):
  File "<stdin>", line 1, in ?
NameError: name 'spam' is not defined
>>> '2' + 2
Traceback (most recent call last):
  File "<stdin>", line 1, in ?
TypeError: Can't convert 'int' object to str implicitly
\end{lstlisting}
错误信息的最后一行指出发生了什么错误。异常也有不同的类型,异常类型做为错误信息的一部分显示出来:示例中的异常分别为 零除错误( \texttt{ZeroDivisionError}) ,命名错误( \texttt{NameError})和类型错误(\texttt{TypeError})。打印错误信息时,异常的类型作为异常的内置名显示。对于所有的内置异常都是如此,不过用户自定义异常就不一定了(尽管这是一个很有用的约定)。标准异常名是内置的标识(没有保留关键字)。

这一行后一部分是关于该异常类型的详细说明,这意味着它的内容依赖于异常类型。

错误信息的前半部分以堆栈的形式列出异常发生的位置。通常在堆栈中列出了源代码行,然而,来自标准输入的源码不会显示出来。

内置的异常 列出了内置异常和它们的含义。
\section{异常处理\label{ErrorException:HandleException}}
通过编程处理选择的异常是可行的。看一下下面的例子:它会一直要求用户输入,直到输入一个合法的整数为止,但允许用户中断这个程序(使用 \texttt{Control-C} 或系统支持的任何方法)。注意:用户产生的中断会引发一个 \texttt{KeyboardInterrupt}异常。
\begin{lstlisting}
>>> while True:
...     try:
...         x = int(input("Please enter a number: "))
...         break
...     except ValueError:
...         print("Oops!  That was no valid number.  Try again...")
...
\end{lstlisting}
\texttt{try}语句按如下方式工作。
\begin{compactitem}
    \item 首先,执行 \texttt{try} 子句 (在 \texttt{try} 和 \texttt{except} 关键字之间的部分)。
    \item 如果没有异常发生, \texttt{except} 子句 在 \texttt{try}语句执行完毕后就被忽略了。
    \item 如果在 \texttt{try} 子句执行过程中发生了异常,那么该子句其余的部分就会被忽略。

    如果异常匹配于 \texttt{except}关键字后面指定的异常类型,就执行对应的 \texttt{except}子句。然后继续执行 \texttt{try}语句之后的代码。
    \item 如果发生了一个异常,在 \texttt{except}子句中没有与之匹配的分支,它就会传递到上一级 \texttt{try}语句中。

    如果最终仍找不到对应的处理语句,它就成为一个\textit{未处理异常},终止程序运行,显示提示信息。
\end{compactitem}
一个 \texttt{try} 语句可能包含多个 \texttt{except} 子句,分别指定处理不同的异常。至多只会有一个分支被执行。异常处理程序只会处理对应的 \texttt{try}子句中发生的异常,在同一个 \texttt{try}语句中,其他子句中发生的异常则不作处理——一个 except 子句可以在括号中列出多个异常的名字,例如:
\begin{lstlisting}
... except (RuntimeError, TypeError, NameError):
...     pass
\end{lstlisting}
最后一个 \texttt{except} 子句可以省略异常名称,以作为通配符使用。你需要慎用此法,因为它会轻易隐藏一个实际的程序错误!可以使用这种方法打印一条错误信息,然后重新抛出异常(允许调用者处理这个异常):
\begin{lstlisting}
import sys

try:
    f = open('myfile.txt')
    s = f.readline()
    i = int(s.strip())
except OSError as err:
    print("OS error: {0}".format(err))
except ValueError:
    print("Could not convert data to an integer.")
except:
    print("Unexpected error:", sys.exc_info()[0])
    raise
\end{lstlisting}
\cprotect\texttt{try \verb|...| except} 语句可以带有一个 \textit{else 子句},该子句只能出现在所有 \texttt{except} 子句之后。当 \texttt{try} 语句没有抛出异常时,需要执行一些代码,可以使用这个子句。例如:
\begin{lstlisting}
for arg in sys.argv[1:]:
    try:
        f = open(arg, 'r')
    except IOError:
        print('cannot open', arg)
    else:
        print(arg, 'has', len(f.readlines()), 'lines')
        f.close()
\end{lstlisting}
使用 \texttt{else} 子句比在 \texttt{try} 子句中附加代码要好,因为这样可以避免 \cprotect\texttt{try \verb|...| except} 意外的截获本来不属于它们保护的那些代码抛出的异常。

发生异常时,可能会有一个附属值,作为异常的 参数 存在。这个参数是否存在、是什么类型,依赖于异常的类型。

在异常名(列表)之后,也可以为\texttt{except}子句指定一个变量。这个变量绑定于一个异常实例,它存储在 \texttt{instance.args} 的参数中。为了方便起见,异常实例定义了 \verb|__str__()| ,这样就可以直接访问过打印参数而不必引用 \texttt{.args}。这种做法不受鼓励。相反,更好的做法是给异常传递一个参数(如果要传递多个参数,可以传递一个元组),把它绑定到 \texttt{message} 属性。一旦异常发生,它会在抛出前绑定所有指定的属性。
\begin{lstlisting}
>>> try:
...    raise Exception('spam', 'eggs')
... except Exception as inst:
...    print(type(inst))    # the exception instance
...    print(inst.args)     # arguments stored in .args
...    print(inst)          # __str__ allows args to be printed directly,
...                         # but may be overridden in exception subclasses
...    x, y = inst.args     # unpack args
...    print('x =', x)
...    print('y =', y)
...
<class 'Exception'>
('spam', 'eggs')
('spam', 'eggs')
x = spam
y = eggs
\end{lstlisting}
对于那些未处理的异常,如果一个它们带有参数,那么就会被作为异常信息的最后部分(“详情”)打印出来。

异常处理器不仅仅处理那些在 \texttt{try} 子句中立刻发生的异常,也会处理那些 \texttt{try} 子句中调用的函数内部发生的异常。例如:
\begin{lstlisting}
>>> def this_fails():
...     x = 1/0
...
>>> try:
...     this_fails()
... except ZeroDivisionError as err:
...     print('Handling run-time error:', err)
...
Handling run-timeerror: int division or modulo by zero
\end{lstlisting}
\section{抛出异常}
\texttt{raise} 语句允许程序员强制抛出一个指定的异常。例如:
\begin{lstlisting}
>>> raise NameError('HiThere')
Traceback (most recent call last):
  File "<stdin>", line 1, in ?
NameError: HiThere
\end{lstlisting}
要抛出的异常由 \texttt{raise} 的唯一参数标识。它必需是一个异常实例或异常类(继承自 \texttt{Exception} 的类)。

如果你需要明确一个异常是否抛出,但不想处理它,\texttt{raise} 语句可以让你很简单的重新抛出该异常:
\begin{lstlisting}
>>> try:
...     raise NameError('HiThere')
... except NameError:
...     print('An exception flew by!')
...     raise
...
An exception flew by!
Traceback (most recent call last):
  File "<stdin>", line 2, in ?
NameError: HiThere
\end{lstlisting}
\section{用户自定义异常}
在程序中可以通过创建新的异常类型来命名自己的异常(Python 类的内容请参见 \Nameref{Class} )。异常类通常应该直接或间接的从 \texttt{Exception} 类派生,例如:
\begin{lstlisting}
>>> class MyError(Exception):
...     def __init__(self, value):
...         self.value = value
...     def __str__(self):
...         return repr(self.value)
...
>>> try:
...     raise MyError(2*2)
... except MyError as e:
...     print('My exception occurred, value:', e.value)
...
My exception occurred, value: 4
>>> raise MyError('oops!')
Traceback (most recent call last):
  File "<stdin>", line 1, in ?
__main__.MyError: 'oops!'
\end{lstlisting}
在这个例子中,\texttt{Exception} 默认的 \verb|__init__()| 被覆盖。新的方式简单的创建 \texttt{value} 属性。这就替换了原来创建 \texttt{args} 属性的方式。

异常类中可以定义任何其它类中可以定义的东西,但是通常为了保持简单,只在其中加入几个属性信息,以供异常处理句柄提取。如果一个新创建的模块中需要抛出几种不同的错误时,一个通常的作法是为该模块定义一个异常基类,然后针对不同的错误类型派生出对应的异常子类:
\begin{lstlisting}
class Error(Exception):
    """Base class for exceptions in this module."""
    pass

class InputError(Error):
    """Exception raised for errors in the input.

    Attributes:
        expression -- input expression in which the error occurred
        message -- explanation of the error
    """

    def __init__(self, expression, message):
        self.expression = expression
        self.message = message

class TransitionError(Error):
    """Raised when an operation attempts a state transition that's not
    allowed.

    Attributes:
        previous -- state at beginning of transition
        next -- attempted new state
        message -- explanation of why the specific transition is not allowed
    """

    def __init__(self, previous, next, message):
        self.previous = previous
        self.next = next
        self.message = message
\end{lstlisting}
与标准异常相似,大多数异常的命名都以“\texttt{Error}” 结尾。

很多标准模块中都定义了自己的异常,用以报告在他们所定义的函数中可能发生的错误。关于类的进一步信息请参见 \Nameref{Class} 一章。
\section{定义清理行为}
\texttt{try} 语句还有另一个可选的子句,目的在于定义在任何情况下都一定要执行的功能。例如:
\begin{lstlisting}
>>> try:
...     raise KeyboardInterrupt
... finally:
...     print('Goodbye, world!')
...
Goodbye, world!
KeyboardInterrupt
Traceback (most recent call last):
  File "<stdin>", line 2, in ?
\end{lstlisting}
不管有没有发生异常, \texttt{finally}子句 在程序离开  \texttt{try} 后都一定会被执行。当  \texttt{try} 语句中发生了未被\texttt{except} 捕获的异常(或者它发生在 \texttt{except} 或 \texttt{else} 子句中),在 \texttt{finally} 子句执行完后它会被重新抛出。 \texttt{try} 语句经由 \texttt{break} ,\texttt{continue} 或 \texttt{return} 语句退 出也一样会执行 \texttt{finally} 子句。以下是一个更复杂些的例子:
\begin{lstlisting}
>>> def divide(x, y):
...     try:
...         result = x / y
...     except ZeroDivisionError:
...         print("division by zero!")
...     else:
...         print("result is", result)
...     finally:
...         print("executing finally clause")
...
>>> divide(2, 1)
result is 2
executing finally clause
>>> divide(2, 0)
division by zero!
executing finally clause
>>> divide("2", "1")
executing finally clause
Traceback (most recent call last):
  File "<stdin>", line 1, in ?
  File "<stdin>", line 3, in divide
TypeError: unsupported operand type(s) for /: 'str' and 'str'
\end{lstlisting}
如你所见, \texttt{finally} 子句在任何情况下都会执行。\texttt{TypeError} 在两个字符串相除的时候抛出,未被 \texttt{except} 子句捕获,因此在 \texttt{finally} 子句执行完毕后重新抛出。

在真实场景的应用程序中,\texttt{finally} 子句用于释放外部资源(文件 或网络连接之类的),无论它们的使用过程中是否出错。
\section{预定义清理行为}
有些对象定义了标准的清理行为,无论对象操作是否成功,不再需要该对象的时候就会起作用。以下示例尝试打开文件并把内容打印到屏幕上。
\begin{lstlisting}
for line in open("myfile.txt"):
    print(line)
\end{lstlisting}
这段代码的问题在于在代码执行完后没有立即关闭打开的文件。这在简单的脚本里没什么,但是大型应用程序就会出问题。\texttt{with} 语句使得文件之类的对象可以 确保总能及时准确地进行清理。
\begin{lstlisting}
with open("myfile.txt") as f:
    for line in f:
        print(line)
\end{lstlisting}
语句执行后,文件 \texttt{f} 总会被关闭,即使是在处理文件中的数据时出错也一样。其它对象是否提供了预定义的清理行为要查看它们的文档。