\chapter{输入和输出\label{InputOutput}}
\markboth{输入和输出}{}
一个程序可以有几种输出方式:以人类可读的方式打印数据,或者写入一个文件供以后使用。本章将讨论几种可能性。
\section{格式化输出}
我们有两种大相径庭地输出值方法:表达式语句 和\texttt{print()}函数(第三种访求是使用文件对象的 \texttt{write()} 方法,标准文件输出可以参考 \texttt{sys.stdout},详细内容参见库参考手册)。

通常,你想要对输出做更多的格式控制,而不是简单的打印使用空格分隔的值。有两种方法可以格式化你的输出:第一种方法是由你自己处理整个字符串,通过使用字符串切割和连接操作可以创建任何你想要的输出形式。\texttt{string} 类型包含一些将字符串填充到指定列宽度的有用操作,随后就会讨论这些。第二种方法是使用 \texttt{str.format()} 方法。

标准模块 \texttt{string} 包括了一些操作,将字符串填充入给定列时,这些操作很有用。随后我们会讨论这部分内容。第二种方法是使用 \texttt{Template}方法。

当然,还有一个问题,如何将值转化为字符串?很幸运,Python 有办法将任意值转为字符串:将它传入 \texttt{repr()} 或 \texttt{str()} 函数。

函数 \texttt{str()} 用于将值转化为适于人阅读的形式,而 \texttt{repr()} 转化为供解释器读取的形式(如果没有等价的语法,则会发生 \texttt{SyntaxError} 异常)某对象没有适于人阅读的解释形式的话,\texttt{str()} 会返回与 \texttt{repr()} 等同的值。很多类型,诸如数值或链表、字典这样的结构,针对各函数都有着统一的解读方式。字符串和浮点数,有着独特的解读方式。

下面有些例子:
\begin{lstlisting}
>>> s = 'Hello, world.'
>>> str(s)
'Hello, world.'
>>> repr(s)
"'Hello, world.'"
>>> str(1/7)
'0.14285714285714285'
>>> x = 10 * 3.25
>>> y = 200 * 200
>>> s = 'The value of x is ' + repr(x) + ', and y is ' + repr(y) + '...'
>>> print(s)
The value of x is 32.5, and y is 40000...
>>> # The repr() of a string adds string quotes and backslashes:
... hello = 'hello, world\n'
>>> hellos = repr(hello)
>>> print(hellos)
'hello, world\n'
>>> # The argument to repr() may be any Python object:
... repr((x, y, ('spam', 'eggs')))
"(32.5, 40000, ('spam', 'eggs'))"
有两种方式可以写平方和立方表:

>>> for x in range(1, 11):
...     print(repr(x).rjust(2), repr(x*x).rjust(3), end=' ')
...     # Note use of 'end' on previous line
...     print(repr(x*x*x).rjust(4))
...
 1   1    1
 2   4    8
 3   9   27
 4  16   64
 5  25  125
 6  36  216
 7  49  343
 8  64  512
 9  81  729
10 100 1000

>>> for x in range(1, 11):
...     print('{0:2d} {1:3d} {2:4d}'.format(x, x*x, x*x*x))
...
 1   1    1
 2   4    8
 3   9   27
 4  16   64
 5  25  125
 6  36  216
 7  49  343
 8  64  512
 9  81  729
10 100 1000
\end{lstlisting}
(注意第一个例子,\texttt{print()}在每列之间加了一个空格,它总是在参数间加入空格。)

以上是一个 \texttt{str.rjust()}方法的演示,它把字符串输出到一列,并通过向左侧填充空格来使其右对齐。类似的方法还有 \texttt{str.ljust()} 和 \texttt{str.center()}。这些函数只是输出新的字符串,并不改变什么。如果输出的字符串太长,它们也不会截断它,而是原样输出,这会使你的输出格式变得混乱,不过总强过另一种选择(截断字符串),因为那样会产生错误的输出值(如果你确实需要截断它,可以使用切割操作,例如:\texttt{x.ljust(n)[:n]} )。

还有另一个方法,\texttt{str.zfill()}它用于向数值的字符串表达左侧填充 0。该函数可以正确理解正负号:
\begin{lstlisting}
>>> '12'.zfill(5)
'00012'
>>> '-3.14'.zfill(7)
'-003.14'
>>> '3.14159265359'.zfill(5)
'3.14159265359'
\end{lstlisting}
方法 \texttt{str.format()}的基本用法如下:
\begin{lstlisting}
>>> print('We are the {} who say "{}!"'.format('knights', 'Ni'))
We are the knights who say "Ni!"
\end{lstlisting}
大括号和其中的字符会被替换成传入 \texttt{str.format()}的参数。大括号中的数值指明使用传入\texttt{str.format()}方法的对象中的哪一个:
\begin{lstlisting}
>>> print('{0} and {1}'.format('spam', 'eggs'))
spam and eggs
>>> print('{1} and {0}'.format('spam', 'eggs'))
eggs and spam
\end{lstlisting}
如果在 \texttt{str.format()} 调用时使用关键字参数,可以通过参数名来引用值:
\begin{lstlisting}
>>> print('This {food} is {adjective}.'.format(
...       food='spam', adjective='absolutely horrible'))
This spam is absolutely horrible.
\end{lstlisting}
位置参数和关键字参数可以随意组合:
\begin{lstlisting}
>>> print('The story of {0}, {1}, and {other}.'.format('Bill', 'Manfred', other='Georg'))
The story of Bill, Manfred, and Georg.
\end{lstlisting}

\texttt{'!a'} (应用 \texttt{ascii()}),\texttt{'!s'} (应用 \texttt{str()})和 \texttt{'!r'} (应用 \texttt{repr()} )可以在格式化之前转换值:
\begin{lstlisting}
>>> import math
>>> print('The value of PI is approximately {}.'.format(math.pi))
The value of PI is approximately 3.14159265359.
>>> print('The value of PI is approximately {!r}.'.format(math.pi))
The value of PI is approximately 3.141592653589793.
\end{lstlisting}
字段名后允许可选的\texttt{':'}和格式指令。这允许对值的格式化加以更深入的控制。下例将 Pi 转为三位精度。
\begin{lstlisting}
>>> import math
>>> print('The value of PI is approximately {0:.3f}.'.format(math.pi))
The value of PI is approximately 3.142.
\end{lstlisting}
在字段后的 \texttt{':'} 后面加一个整数会限定该字段的最小宽度,这在美化表格时很有用:
\begin{lstlisting}
>>> table = {'Sjoerd': 4127, 'Jack': 4098, 'Dcab': 7678}
>>> for name, phone in table.items():
...     print('{0:10} ==> {1:10d}'.format(name, phone))
...
Jack       ==>       4098
Dcab       ==>       7678
Sjoerd     ==>       4127
\end{lstlisting}
如果你有个实在是很长的格式化字符串,不想分割它。如果你可以用命名来引用被格式化的变量而不是位置就好了。有个简单的方法,可以传入一个字典,用中括号( \verb|'[]'| )访问它的键:
\begin{lstlisting}
>>> table = {'Sjoerd': 4127, 'Jack': 4098, 'Dcab': 8637678}
>>> print('Jack: {0[Jack]:d}; Sjoerd: {0[Sjoerd]:d}; '
          'Dcab: {0[Dcab]:d}'.format(table))
Jack: 4098; Sjoerd: 4127; Dcab: 8637678
\end{lstlisting}
也可以用\verb|'**'|标志将这个字典以关键字参数的方式传入:
\begin{lstlisting}
>>> table = {'Sjoerd': 4127, 'Jack': 4098, 'Dcab': 8637678}
>>> print('Jack: {Jack:d}; Sjoerd: {Sjoerd:d}; Dcab: {Dcab:d}'.format(**table))
Jack: 4098; Sjoerd: 4127; Dcab: 8637678
\end{lstlisting}
这种方式与新的内置函数 \texttt{vars()}组合使用非常有效。该函数返回包含所有局部变量的字典。

要进一步了解字符串格式化方法 \texttt{str.format()},参见 格式字符串语法。
\subsection{旧式的字符串格式化}
操作符\verb|%|也可以用于字符串格式化。它以类似 \texttt{sprintf()}-style 的方式解析左参数,将右参数应用于此,得到格式化操作生成的字符串,例如:
\begin{lstlisting}
>>> import math
>>> print('The value of PI is approximately %5.3f.' % math.pi)
The value of PI is approximately 3.142.
\end{lstlisting}
更多的信息可以参见 printf-style String Formatting 一节。
\section{文件读写}
函数 \texttt{open()}返回\href{https://docs.python.org/3/glossary.html#term-file-object}{\textit{文件对象}},通常的用法需要两个参数:\texttt{open(filename, mode)}。
\begin{lstlisting}
>>> f = open('workfile', 'w')
\end{lstlisting}
第一个参数是一个含有文件名的字符串。第二个参数也是一个字符串,含有描述如何使用该文件的几个字符。\textit{mode}为 ‘\verb|r|’ 时表示只是读取文件;‘\verb|w|’ 表示只是写入文件(已经存在的同名文件将被删掉);‘\verb|a|’ 表示打开文件进行追加,写入到文件中的任何数据将自动添加到末尾。 ‘\verb|r+|’ 表示打开文件进行读取和写入。\textit{mode}参数是可选的,默认为 ‘\verb|r|’。

通常,文件以 \textit{文本模式} 打开,这意味着,你从文件读出和向文件写入的字符串会被特定的编码方式(默认是UTF-8)编码。模式后面的 ‘\verb|b|’ 以 \textit{二进制模式}打开文件:数据会以字节对象的形式读出和写入。这种模式应该用于所有不包含文本的文件。

在文本模式下,读取时默认会将平台有关的行结束符(Unix上是 \verb|\n| , Windows上是 \verb|\r\n|)转换为 \verb|\n|。在文本模式下写入时,默认会将出现的 \verb|\n| 转换成平台有关的行结束符。这种暗地里的修改对 ASCII 文本文件没有问题,但会损坏 \verb|JPEG| 或 \verb|EXE| 这样的二进制文件中的数据。使用二进制模式读写此类文件时要特别小心。
\subsection{文件对象方法}
本节中的示例都默认文件对象 \texttt{f} 已经创建。

要读取文件内容,需要调用 \texttt{f.read(size)},该方法读取若干数量的数据并以字符串形式返回其内容,\textit{size} 是可选的数值,指定字符串长度。如果没有指定 \textit{size} 或者指定为负数,就会读取并返回整个文件。当文件大小为当前机器内存两倍时,就会产生问题。反之,会尽可能按比较大的 \textit{size} 读取和返回数据。如果到了文件末尾,\texttt{f.read()}会返回一个空字符串(\verb|''|):
\begin{lstlisting}
>>> f.read()
'This is the entire file.\n'
>>> f.read()
''
\end{lstlisting}

\texttt{f.readline()} 从文件中读取单独一行,字符串结尾会自动加上一个换行符(\verb|\n| ),只有当文件最后一行没有以换行符结尾时,这一操作才会被忽略。这样返回值就不会有混淆,如果 \texttt{f.readline()} 返回一个空字符串,那就表示到达了文件末尾,如果是一个空行,就会描述为 ‘\verb|\n|’,一个只包含换行符的字符串:
\begin{lstlisting}
>>> f.readline()
'This is the first line of the file.\n'
>>> f.readline()
'Second line of the file\n'
>>> f.readline()
''
\end{lstlisting}
你可以循环遍历文件对象来读取文件中的每一行。这是一种内存高效、快速,并且代码简介的方式:
\begin{lstlisting}
>>> for line in f:
...     print(line, end='')
...
This is the first line of the file.
Second line of the file
\end{lstlisting}
如果你想把文件中的所有行读到一个列表中,你也可以使用 \texttt{list(f)} 或者 \texttt{f.readlines()}。

\texttt{f.write(string)} 方法将 \textit{string} 的内容写入文件,并返回写入字符的长度:
\begin{lstlisting}
>>> f.write('This is a test\n')
15
\end{lstlisting}
想要写入其他非字符串内容,首先要将它转换为字符串:
\begin{lstlisting}
>>> value = ('the answer', 42)
>>> s = str(value)
>>> f.write(s)
18
\end{lstlisting}
\texttt{f.tell()} 返回一个整数,代表文件对象在文件中的指针位置,该数值计量了自文件开头到指针处的比特数。需要改变文件对象指针话话,使用 \cprotect\texttt{f.seek(offset,\verb|from_what|)}。指针在该操作中从指定的引用位置移动 \textit{offset} 比特,引用位置由 \textit{from\_what} 参数指定。 \textit{from\_what} 值为 0 表示自文件起始处开始,1 表示自当前文件指针位置开始,2 表示自文件末尾开始。\textit{from\_what} 可以忽略,其默认值为零,此时从文件头开始:
\begin{lstlisting}
>>> f = open('workfile', 'rb+')
>>> f.write(b'0123456789abcdef')
16
>>> f.seek(5)     # Go to the 6th byte in the file
5
>>> f.read(1)
b'5'
>>> f.seek(-3, 2) # Go to the 3rd byte before the end
13
>>> f.read(1)
b'd'
\end{lstlisting}
在文本文件中(没有以 \verb|b| 模式打开),只允许从文件头开始寻找(有个例外是用 \texttt{seek(0, 2)} 寻找文件的最末尾处)而且合法的 \textit{offset} 值只能是 \texttt{f.tell()} 返回的值或者是零。其它任何 \textit{offset} 值都会产生未定义的行为。

当你使用完一个文件时,调用 \texttt{f.close()}方法就可以关闭它并释放其占用的所有系统资源。 在调用 \texttt{f.close()}方法后,试图再次使用文件对象将会自动失败。
\begin{lstlisting}
>>> f.close()
>>> f.read()
Traceback (most recent call last):
  File "<stdin>", line 1, in ?
ValueError: I/O operation on closed file
\end{lstlisting}
用关键字 \texttt{with} 处理文件对象是个好习惯。它的先进之处在于文件用完后会自动关闭,就算发生异常也没关系。它是 \texttt{try-finally} 块的简写:
\begin{lstlisting}
>>> with open('workfile', 'r') as f:
...     read_data = f.read()
>>> f.closed
True
\end{lstlisting}
文件对象还有一些不太常用的附加方法,比如 \texttt{isatty()}和 \texttt{truncate()}在库参考手册中有文件对象的完整指南。
\subsection{使用 \texttt{json} 存储结构化数据}
从文件中读写字符串很容易。数值就要多费点儿周折,因为 \texttt{read()} 方法只会返回字符串,应将其传入 \texttt{int()} 这样的函数,就可以将 \verb|'123'| 这样的字符串转换为对应的数值 123。当你想要保存更为复杂的数据类型,例如嵌套的列表和字典,手工解析和序列化它们将变得更复杂。

好在用户不是非得自己编写和调试保存复杂数据类型的代码,Python 允许你使用常用的数据交换格式 \href{http://json.org/}{\textit{JSON(JavaScript Object Notation)}}。标准模块 \texttt{json} 可以接受 Python 数据结构,并将它们转换为字符串表示形式;此过程称为 \textit{序列化}。从字符串表示形式重新构建数据结构称为 \textit{反序列化}。序列化和反序列化的过程中,表示该对象的字符串可以存储在文件或数据中,也可以通过网络连接传送给远程的机器。
\begin{Warning}{注意}
JSON 格式经常用于现代应用程序中进行数据交换。许多程序员都已经熟悉它了,使它成为相互协作的一个不错的选择。
\end{Warning}
如果你有一个对象 \verb|x|,你可以用简单的一行代码查看其 JSON 字符串表示形式:
\begin{lstlisting}
>>> json.dumps([1, 'simple', 'list'])
'[1, "simple", "list"]'
\end{lstlisting}
\texttt{dumps()} 函数的另外一个变体 \texttt{dump()},直接将对象序列化到一个文件。所以如果 \texttt{f} 是为写入而打开的一个 文件对象,我们可以这样做:
\begin{lstlisting}
json.dump(x, f)
\end{lstlisting}
为了重新解码对象,如果 \texttt{f} 是为读取而打开的 文件对象:
\begin{lstlisting}
x = json.load(f)
\end{lstlisting}
这种简单的序列化技术可以处理列表和字典,但序列化任意类实例为 JSON 需要一点额外的努力。 \texttt{json}模块的手册对此有详细的解释。\\
\textbf{See also:}\\
\verb|pickle|-pickle 模块

与 JSON 不同,\textit{pickle} 是一个协议,它允许任意复杂的 Python 对象的序列化。因此,它只能用于 Python 而不能用来与其他语言编写的应用程序进行通信。默认情况下它也是不安全的:如果数据由熟练的攻击者精心设计, 反序列化来自一个不受信任源的 pickle 数据可以执行任意代码。