\chapter{开胃菜\label{Appetite}}
\markboth{开胃菜}{}
如果你要用计算机做很多工作,最后你会发现有一些任务你更希望用自动化的方式进行处理。比如,你想要在大量的文本文件中执行查找/替换,或者以复杂的方式对大量的图片进行重命名和整理。也许你想要编写一个小型的自定义数据库、一个特殊的 GUI 应用程序或一个简单的小游戏。\

如果你是一名专业的软件开发者,可能你必须使用几种 C/C++/JAVA 类库,并且发现通常编写/编译/测试/重新编译的周期是如此漫长。也许你正在为这些类库编写测试用例,但是发现这是一个让人烦躁的工作。又或者你已经完成了一个可以使用扩展语言的程序,但你并不想为此重新设计并实现一套全新的语言。

那么 Python 正是你所需要的语言。

虽然你能够通过编写 Unix shell 脚本或 Windows 批处理文件来处理其中的某些任务,但 Shell 脚本更适合移动文件或修改文本数据,并不适合编写 GUI 应用程序或游戏;虽然你能够使用 C/C++/JAVA 编写程序,但即使编写一个简单的 first-draft 程序也有可能耗费大量的开发时间。相比之下,Python 更易于使用,无论在 Windows、Mac OS X 或 Unix 操作系统上它都会帮助你更快地完成任务。

虽然 Python 易于使用,但它却是一门完整的编程语言;与 Shell 脚本或批处理文件相比,它为编写大型程序提供了更多的结构和支持。另一方面,Python 提供了比 C 更多的错误检查,并且作为一门 \emph{高级语言},它内置支持高级的数据结构类型,例如:灵活的数组和字典。因其更多的通用数据类型,Python 比 Awk 甚至 Perl 都适用于更多问题领域,至少大多数事情在 Python 中与其他语言同样简单。

Python 允许你将程序分割为不同的模块,以便在其他的 Python 程序中重用。Python 内置提供了大量的标准模块,你可以将其用作程序的基础,或者作为学习 Python 编程的示例。这些模块提供了诸如文件 I/O、系统调用、Socket 支持,甚至类似 Tk 的用户图形界面(GUI)工具包接口。

Python 是一门解释型语言,因为无需编译和链接,你可以在程序开发中节省宝贵的时间。Python 解释器可以交互的使用,这使得试验语言的特性、编写临时程序或在自底向上的程序开发中测试方法非常容易。你甚至还可以把它当做一个桌面计算器。

Python 让程序编写的紧凑和可读。用 Python 编写的程序通常比同样的 C、C++ 或 Java 程序更短小,这是因为以下几个原因:
\begin{compactitem}
  \item 高级数据结构使你可以在一条语句中表达复杂的操作;
  \item 语句组使用缩进代替开始和结束大括号来组织;
  \item 变量或参数无需声明。
\end{compactitem}

Python 是 \emph{可扩展的}:如果你会 C 语言编程便可以轻易地为解释器添加内置函数或模块,或者为了对性能瓶颈作优化,或者将 Python 程序与只有二进制形式的库(比如某个专业的商业图形库)连接起来。一旦你真正掌握了它,你可以将 Python 解释器集成进某个 C 应用程序,并把它当作那个程序的扩展或命令行语言。

顺便说一句,这个语言的名字来自于 BBC 的 “Monty Python's Flying Cirecus” 节目,和爬行类动物没有任何关系。在文档中引用 Monty Python 的典故不仅可行,而且值得鼓励!

现在你已经为 Python 兴奋不已了吧,大概想要领略一些更多的细节!学习一门语言最好的方法就是使用它,本指南推荐你边读边使用 Python 解释器练习。

下一节中,我们将解释 Python 解释器的用法。这是很简单的一件事情,但它有助于试验后面的例子。

本手册剩下的部分将通过示例介绍 Python 语言及系统的诸多特性,开始是简单的语法、数据类型和表达式,接着介绍函数与模块,最后涉及异常和自定义类这样的高级内容。
