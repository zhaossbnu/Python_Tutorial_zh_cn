\chapter{交互式输入行编辑历史回顾\label{EditAndHistory}}
\markboth{交互式输入行编辑历史回顾}{}
某些版本的 Python 解释器支持编辑当前的输入行和历史记录,类似于在 Korn shell 和 GNU Bash shell 中看到的功能。这是使用 \href{https://tiswww.case.edu/php/chet/readline/rltop.html}{GNU Readline} 库实现的,它支持各种编辑风格。 这个库有它自己的文档,在这里我们不就重复了。
\section{Tab补全和历史回顾}
变量和模块名的补全在解释器启动时 自动打开 以便 \texttt{Tab} 键调用补全功能;它会查看Python语句的名字,当前局部变量以及可以访问的模块名。对于点分表达式如 \texttt{string.a},它将求出表达式最后一个 '.' 之前的值,然后根据结果的属性给出补全的建议。注意,如果一个具有 \verb|__getattr__()| 方法的对象是表达式的某部分,这可能执行应用程序定义的代码。默认的配置同时会把历史记录保存在你的用户目录下一个名为 \verb|.python_history| 的文件中。在下次与交互式解释器的回话中,历史记录将还可以访问。
\section{其他交互式解释器}
与早期版本的解释器相比,现在是向前巨大的进步;然而,有些愿望还是没有实现:如果能对连续的行给出正确的建议就更好了(解析器知道下一行是否需要缩进)。补全机制可以使用解释器的符号表。检查(或者只是建议)匹配的括号、 引号的命令等也会非常有用。

一个增强的交互式解释器是 \href{https://ipython.org/}{IPython},它已经存在相当一段时间,具有 tab 补全、 对象 exploration 和高级的历史记录功能。它也可以彻底定制并嵌入到其他应用程序中。另一个类似的增强的交互式环境是 \href{https://www.bpython-interpreter.org}{bpython}。 